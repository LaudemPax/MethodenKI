\documentclass{wissdoc}
%\documentclass[oneside]{wissdoc}
% ----------------------------------------------------------------
% Diplomarbeit - Hauptdokument
% ----------------------------------------------------------------
% wissdoc Optionen: draft, relaxed, pdf, oneside --> siehe wissdoc.cls
% ------------------------------------------------------------------
% Packages für Deckblatt
\usepackage[absolute]{textpos} 	%Textboxen an absolute Position setzen
\usepackage{setspace}						%Zeilenabstand anpassen
\usepackage{color}							%Farbige Schrift
\usepackage{graphicx}						%Einbinden von Grafiken

% Weitere packages: (Dokumentation dazu durch "latex <package>.dtx")
% \usepackage{varioref}
% \usepackage{verbatim}
% \usepackage{float}    %z.B. \floatstyle{ruled}\restylefloat{figure}
\usepackage{subfigure}
\usepackage[ngerman]{babel}
\usepackage[T1]{fontenc}
\usepackage[ansinew]{inputenc}
\usepackage{tabularx}

\usepackage{float}

% ----------------------------------------------------------------------
% used for code examples
\usepackage{listings}

\definecolor{lightgray}{rgb}{.9,.9,.9}
\definecolor{darkgray}{rgb}{.4,.4,.4}
\definecolor{purple}{rgb}{0.65, 0.12, 0.82}


\renewcommand\lstlistingname{Quelltext} % Change language of section name
\usepackage{xcolor}
\lstset{ % General setup for the package
	language=bash,
	basicstyle=\small\sffamily,
	backgroundcolor=\color{lightgray},
	numbers=left,
	numberstyle=\tiny,
	frame=tb,
	tabsize=4,
	columns=fixed,
	showstringspaces=false,
	showtabs=false,
	keepspaces,
	commentstyle=\color{red},
	keywordstyle=\color{blue}
}


\lstdefinelanguage{JavaScript}{
	keywords={typeof, new, true, false, catch, function, return, null, catch, switch, var, if, in, while, do, else, case, break},
	keywordstyle=\color{blue}\bfseries,
	ndkeywords={class, export, boolean, throw, implements, import, this},
	ndkeywordstyle=\color{darkgray}\bfseries,
	identifierstyle=\color{black},
	sensitive=false,
	comment=[l]{//},
	morecomment=[s]{/*}{*/},
	commentstyle=\color{purple}\ttfamily,
	stringstyle=\color{red}\ttfamily,
	morestring=[b]',
	morestring=[b]"
}

\lstset{
	language=JavaScript,
	backgroundcolor=\color{lightgray},
	extendedchars=true,
	basicstyle=\footnotesize\ttfamily,
	showstringspaces=false,
	showspaces=false,
	numbers=left,
	numberstyle=\footnotesize,
	numbersep=9pt,
	tabsize=2,
	breaklines=true,
	showtabs=false,
	captionpos=b
}
% ----------------------------------------------------------------------

% Zeilenabstand nach Vorgabe - Falls gefordert
%\setstretch{1,3} 

% Inhaltsangabe auf Unterabschnitte(2 Ebenen) begrenzen
\setcounter{tocdepth}{2}


% \usepackage{color}    % Farbiger/grauer Text
% \usepackage{colortbl}   % Farbige/graue Tabellenzeilen und -spalten!! <--
% \usepackage{fancybox} % für schattierte,ovale Boxen etc.
% \usepackage{tabularx} % automatische Spaltenbreite
% \usepackage{supertab} % mehrseitige Tabellen
%% ---------------- end of usepackages -------------

%% Informationen für die PDF-Datei
\hypersetup{pdfauthor={Max Mustermann},%
            pdftitle={Bachelorarbeit},%
            pdfsubject={Titel der Arbeit},%
            pdfkeywords={Forschung, Entwicklung, Funktechnik},%
            pdfproducer={LaTeX},%
            pdfcreator={pdfLaTeX}
}

% Macros, nicht unbedingt notwendig
\input{macros}

% Print URLs not in Typewriter Font
\def\UrlFont{\rm}

\newcommand{\blankpage}{% Leerseite ohne Seitennummer, nächste Seite rechts
 \clearpage{\pagestyle{empty}\cleardoublepage}
}

%% Einstellungen für das gesamte Dokument

% Trennhilfen
% Wichtig!
% Im german-paket sind zusätzlich folgende Trennhinweise enthalten:
% "- = zusätzliche Trennstelle
% "| = Vermeidung von Ligaturen und mögliche Trennung (bsp: Schaf"|fell)
% "~ = Bindestrich an dem keine Trennung erlaubt ist (bsp: bergauf und "~ab)
% "= = Bindestrich bei dem Worte vor und dahinter getrennt werden dürfen
% "" = Trennstelle ohne Erzeugung eines Trennstrichs (bsp: und/""oder)

% Trennhinweise fuer Woerter hier beschreiben
\hyphenation{
% Pro-to-koll-in-stan-zen
% Ma-na-ge-ment  Netz-werk-ele-men-ten
% Netz-werk Netz-werk-re-ser-vie-rung
% Netz-werk-adap-ter Fein-ju-stier-ung
% Da-ten-strom-spe-zi-fi-ka-tion Pa-ket-rumpf
% Kon-troll-in-stanz
}

%Tabellen Kommandos
\newcolumntype{L}[1]{>{\raggedright\arraybackslash}p{#1}}
\newcolumntype{C}[1]{>{\centering\arraybackslash}p{#1}}
\newcolumntype{R}[1]{>{\raggedleft\arraybackslash}p{#1}}

% Index-Datei öffnen
\ifnotdraft{\makeindex}
%%%%%%%%%%%%%% includeonly %%%%%%%%%%%%%%%%%%%
% Es werden nur die Teile eingebunden, die hier aufgefuehrt sind!
%\includeonly{%
%titelseite,%
%erklaerung,%
%kurzfassung,%
%einleitung,%
%analyse,%
%entwurf,%
%implemen,%
%zusammenf%
%}
%%%%%%%%%%%%%%%%%%%%%%%%%%%%%%%%%%%%%%%%%%%%%%
\begin{document}
%Auskommentiert, da nicht notwendig für das Praktikum
%\ifnotdraft{
	%%%%Vorlage
	% 
\thispagestyle{empty}
\begin{figure}[t]
		\hspace{-2.2cm}\includegraphics[width=0.93\paperwidth]{figures/deckblatt_prax_bac}
	\label{fig:deckblatt_prax_bac}
\end{figure}
  %<-- Nach Vorgabe der HS Augsburg

	\include{deckblatt}  %<-- Nach Vorgabe der HS Augsburg
	%
	%%%% Innere Titelseite 
 	%\include{titelseite} %<-- Vorgabe Prüfer oder frei wählbar
	%
	%%%%Optional - Falls von der Firma gefordert
	%\include{sperrvermerk}
	%
	%%%%Pflicht
 	%\include{erklaerung}
	%
	%%% Leere Seite bei zweiseitigem Druck
	%\ifnotonesideelse{\blankpage}{}
	%\include{kurzfassung}
	%%% Leere Seite bei zweiseitigem Druck
	%\ifnotonesideelse{\blankpage}{}
%}



%
%% ++++++++++++++++++++++++++++++++++++++++++
%% Verzeichnisse
%% ++++++++++++++++++++++++++++++++++++++++++
\pagenumbering{roman}
\ifnotdraft{
\tableofcontents
% Leere Seite bei zweiseitigem Druck
%\ifnotonesideelse{\blankpage}{}
%\listoffigures
%% Leere Seite bei zweiseitigem Druck
%\ifnotonesideelse{\blankpage}{}
%\listoftables
%% Leere Seite bei zweiseitigem Druck
%\ifnotonesideelse{\blankpage}{}
}
%% ++++++++++++++++++++++++++++++++++++++++++
%% Hauptteil
%% ++++++++++++++++++++++++++++++++++++++++++
\graphicspath{{figures/}}
\pagenumbering{arabic}

%%% Ab hier eigene Kapitel einfügen
%%% Kapitel sind analog zur Wordvorlage zu wählen

\chapter{Introduction}

\section{Was ist k�nstliche Intelligenz?}

Es gibt viele Beispiele daf�r, dass KI in der realen Welt aktiv ist, insbesondere mit der heutigen Technologie. Von relativ einfachen Anwendungen wie Computergegnern in rundenbasierten Spielen, �ber n�tzliche Expertensysteme bis hin zu selbstfahrenden Autos und KI-generierter Kunst. 

Doch auch heute noch ist es f�r Forscher und Experten schwierig, sich auf eine gute Definition von KI zu einigen. Versuche, KI zu definieren, scheinen entweder irrelevant zu werden, wenn sich die Technologie weiterentwickelt und als trivial angesehen wird, oder sie decken nicht das breite Spektrum der Anwendungen ab, f�r die KI eingesetzt wird.

Im Allgemeinen lassen sich die Anwendungen der KI in die folgenden drei Arten unterteilen:
\begin{itemize}
    \item \textbf{Artificial Narrow Intelligence}: Eine KI, die sich auf eine bestimmte Aufgabe spezialisiert hat und diese Aufgabe besser erledigt als ein Mensch. Diese KI ist aber nicht auf andere Aufgaben anwendbar, f�r die sie nicht definiert ist.
    \item \textbf{Artificial General Intelligence}: Eine KI, die in der Lage ist, die Welt zu beobachten und aus diesen Beobachtungen Schlussfolgerungen zu ziehen, wie es ein Mensch tun w�rde. Es besteht Einigkeit dar�ber, dass diese Stufe der KI noch nicht erreicht ist, aber man ist sich nicht einig, wann wir sie erreichen werden oder ob sie �berhaupt m�glich ist.  
    \item \textbf{Artificial Super Intelligence}: Eine KI, die dem Menschen in jeder Hinsicht �berlegen ist. Derzeit Science-Fiction, wirft aber ethische Fragen auf.
\end{itemize}

Das Problem bei der Definition von k�nstlicher Intelligenz beginnt wahrscheinlich damit, dass es keine einheitliche Definition dessen gibt, was ``nat�rliche Intelligenz'' ist. Dies wird in philosophischen und psychologischen Kreisen heftig diskutiert.

Zusammenfassend l�sst sich sagen, dass es keine einheitliche Definition von KI gibt und dass die einzigen n�tzlichen Anwendungen von KI, die wir heute haben, als die grundlegendste eingestuft werden, n�mlich als ``Artificial Narrow Intelligence''.

\section{Stand der Technik}

Heutzutage werden KI-Techniken in der Industrie stark genutzt, meist in Form von Komponenten eines gr��eren, komplexeren Systems. Das bedeutet, dass die Technologie in den meisten F�llen ausgereift genug ist, um n�tzlich zu sein, und dass sie f�r Unternehmen skalierbar ist. 
 
Verschiedene Arten von KI kommen in vielen unterschiedlichen Bereichen zum Einsatz, z. B. bei Expertensystemen, Data Mining, nat�rlicher Sprachverarbeitung, Bilderkennung, KI in Spielen, selbstfahrenden Robotern und vielen, vielen mehr.

Es ist jedoch noch ein langer Weg zu gehen, da es noch viele ungel�ste Probleme in der KI gibt. Derzeit ist die meiste KI spezifisch und emuliert nur menschliches Verhalten. Es ist immer noch nicht m�glich, eine KI zu schaffen, die z. B. jedes beliebige Spiel erlernen und spielen oder ihre Intelligenz f�r andere Aufgaben nutzen kann, f�r die sie nicht entwickelt wurde. 

Dennoch ist KI ein altes Feld mit vielen Techniken und Anwendungsmethoden. Dieses Portfolio fasst diese Methoden und Techniken zusammen und untersucht ihre Eigenschaften.

\input{kapitel/1/kap_1.3.tex}
\chapter{Formulierung von Problemen und Lösungen in der Symbolischen Informationsverarbeitung}
% \include{einleitung}
% \include{unternehmen}

% \include{projekt_a}
% \include{stellungnahme}

%\include{beispiele}   % Beispiele
%\include{beispiele2}     % Beispiele2

%\chapter*{Hinweis Literatur}
Beachten Sie, dass auch in einem Praxisbericht alle verwendeten Quellen eindeutig gekennzeichnet sein m�ssen. Insbesondere m�ssen Sie auch Bildquellen angeben (sofern Sie die Bilder nicht selbst aufgenommen/erstellt haben). Ebenso muss die Verwendung von Inhalten aus Firmenpr�sentationen angegeben werden. Der Bezug der Textstelle zur Quelle muss eindeutig sein.

Vermeiden Sie die Angabe von Webseiten als Quellen. Wenn Sie diese dennoch verwenden wollen, achten Sie auf eine Angabe der URL mit Abrufdatum und erg�nzen Sie die Quelle falls m�glich mit Autor und Titel.


%% ++++++++++++++++++++++++++++++++++++++++++
%% Anhang
%% ++++++++++++++++++++++++++++++++++++++++++

%\appendix
%\include{anhang_a}
%\include{anhang_b}

%\ifnotonesideelse{\cleardoublepage}{}

%% ++++++++++++++++++++++++++++++++++++++++++
%% Literatur
%% ++++++++++++++++++++++++++++++++++++++++++
\addcontentsline{toc}{chapter}{\bibname}
%  mit dem Befehl \nocite werden auch nicht zitierte Referenzen abgedruckt 
% (normalerweise nicht erwünscht)
% \nocite{*}
\bibliographystyle{rialpha}
%Einbinden Bibtexdatei - Direkt aus JabRef generiert
\bibliography{literatur}
%% ++++++++++++++++++++++++++++++++++++++++++
%% Index (optional)
%% ++++++++++++++++++++++++++++++++++++++++++
%\ifnotdraft{
%\addcontentsline{toc}{chapter}{Index}
%\printindex            % Index, Stichwortverzeichnis
%}
\end{document}
