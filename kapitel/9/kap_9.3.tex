\section{STRIPS-Planer}

STRIPS steht f�r ``Stanford Research Institute Problem Solver''. In STRIPS wird eine Planungsaufgabe als \((S, O, F)\) dargestellt, wobei \(S\) und \(F\) eine Menge von Pr�dikaten sind. \(S\) beschreibt den Startzustand, \(F\) den Zielzustand und \(O\) ist eine Menge von STRIPS-Operatorschemata.

Ein STRIPS-Planoperator ist der Tripel \(Op = (P, D, A)\). \(P\) sind die Vorbedingungen (engl. Preconditions) die gelten m�ssen, damit Op auf eine Konstellation anwendbar ist. \(D\) ist die Menge der wegfallenden Fakten (Delete List) nach einer Ausf�hrung von Op. \(A\) ist die Menge der neuen Fakten (Add List), die nach einer Ausf�hrung von Op gelten.

\paragraph{Beispiele f�r eine Operator-Schema mit Variablen}

Ausf�hrung von \(pickup(x)\), einen Block vom Tisch aufheben:

\begin{itemize}
    \item \(P:{OnTable(x), Clear(x), Handempty}\)
    \item \(D:{Holding(x), Clear(y), Handempty}\)
    \item \(A:{Holding(x)}\)
\end{itemize}

Ausf�hrung von \(putdown(x)\), einen Block, den der Greifer gerade h�lt, auf den Tisch legen:

\begin{itemize}
    \item \(P:{Holding(x)}\)
    \item \(D:{Holding(x)}\)
    \item \(A:{OnTable(x), Clear(x), Handempty}\)
\end{itemize}

Ausf�hrung von \(stack(x, y)\), was bedeutet, dass Block X auf Block Y gestapelt wird:

\begin{itemize}
    \item \(P:{Holding(x), Clear(y)}\)
    \item \(D:{Holding(x), Clear(y)}\)
    \item \(A:{Handempty, Clear(x), On(x,y)}\)
\end{itemize}

Ausf�hrung von \(unstack(x, y)\), was bedeutet, dass Block x von der Oberseite von Block y aufgenommen wird:

\begin{itemize}
    \item \(P:{Handempty, Clear(x), On(x,y)}\)
    \item \(D:{Handempty, Clear(x), On(x,y)}\)
    \item \(A:{Holding(x), Clear(y)}\)
\end{itemize}

\paragraph{Ausf�hrung einer Aktion mittels STRIPS}

\begin{enumerate}
    \item Unifiziere der Vorbedingungen der Aktion mit der aktuellen Zustandsbeschreibung.
    \item Instantiiere die Add- und Delete-Listen mit dem gefundenen Unifikator.
    \item L�sche die Atome der instantiierten Delete-Liste aus der aktuellen Zustandsbeschreibung.
    \item F�ge die Atome der Add-Liste der neuen Zustandsbeschreibung hinzu.
\end{enumerate}