\section{Vertiefung: PDDL verstehen und ausf�hren}

Das 1-dimensionale Rubiks-Cube-Problem von hakank.org\cite{pddl-hakank} wurde als Beispiel verwendet. In dieser Aufgabe sind die Zahlen 1 bis 6 in einer bestimmten Reihenfolge angegeben. Das Ziel der Aufgabe ist es, diese Zahlen in zunehmender Reihenfolge zu sortieren (also 1,2,3,4,5,6), wobei drei m�gliche 4-stellige Drehungen verwendet werden k�nnen:-
\begin{itemize}
    \item \textbf{rot1:} (1,2,3,4),5,6 \(\rightarrow\) (4,3,2,1),5,6.
    \item \textbf{rot2:} 1,(2,3,4,5),6 \(\rightarrow\) 1,(5,4,3,2),6
    \item \textbf{rot3:} 1,2,(3,4,5,6) \(\rightarrow\) 1,2,(6,5,4,3)
\end{itemize}

Die Dom�ne und ein Beispielproblem wurden mit PDDL kodiert und ein Online-Planer\cite{pddl-online} wurde verwendet.

\paragraph{Domain definition:}
\begin{lstlisting}
(define (domain rubik-1d)
(:requirements :strips)
(:predicates (pos1 ?v)
                (pos2 ?v)
                (pos3 ?v)
                (pos4 ?v)
                (pos5 ?v)
                (pos6 ?v))

;     _______                _______
;    (1 2 3 4)5 6  --(rot0)->  (4 3 2 1)5 6
;
(:action rot0
            :parameters (?v1 ?v2 ?v3 ?v4 ?v5 ?v6)
            :precondition (and (pos1 ?v1) 
                                (pos2 ?v2) 
                                (pos3 ?v3) 
                                (pos4 ?v4) 
                                (pos5 ?v5) 
                                (pos6 ?v6) 
                                )
            :effect  (and 
                        (pos1 ?v4)              
                        (pos2 ?v3)              
                        (pos3 ?v2)              
                        (pos4 ?v1)              
                        (pos5 ?v5)              
                        (pos6 ?v6)      
                        (not (pos1 ?v1))
                        (not (pos2 ?v2))
                        (not (pos3 ?v3))
                        (not (pos4 ?v4)))
    ) 

    ;
    ;       _______                   _______
    ;     1(2 3 4 5)6  --(rot1)->   1(5 4 3 2)6
    ;
(:action rot1
            :parameters (?v1 ?v2 ?v3 ?v4 ?v5 ?v6)
            :precondition (and (pos1 ?v1) 
                                (pos2 ?v2) 
                                (pos3 ?v3) 
                                (pos4 ?v4) 
                                (pos5 ?v5) 
                                (pos6 ?v6) 
                                )
            :effect  (and 
                        (pos1 ?v1)
                        (pos2 ?v5)              
                        (pos3 ?v4)              
                        (pos4 ?v3)              
                        (pos5 ?v2)
                        (pos6 ?v6)
                        (not (pos2 ?v2))
                        (not (pos3 ?v3))
                        (not (pos4 ?v4))
                        (not (pos5 ?v5)))
    ) 

    ;
    ;      _______                   _______
    ;  1 2(3 4 5 6) --(rot2)->   1 2(6 5 4 3)
    ;
    (:action rot2
            :parameters (?v1 ?v2 ?v3 ?v4 ?v5 ?v6)
            :precondition (and (pos1 ?v1) 
                                (pos2 ?v2) 
                                (pos3 ?v3) 
                                (pos4 ?v4) 
                                (pos5 ?v5) 
                                (pos6 ?v6) 
                                )
            :effect  (and 
                        (pos1 ?v1)
                        (pos2 ?v2)              
                        (pos3 ?v6)              
                        (pos4 ?v5)              
                        (pos5 ?v4)              
                        (pos6 ?v3)              
                        (not (pos3 ?v3))
                        (not (pos4 ?v4))
                        (not (pos5 ?v5))
                        (not (pos6 ?v6)))
    ) 
)
\end{lstlisting}

\paragraph{Problem definition:}
\begin{lstlisting}
(define (problem rubik-01)
(:domain rubik-1d)
(:objects v1 v2 v3 v4 v5 v6)
(:init
    (pos1 v1)
    (pos2 v3)
    (pos3 v2)
    (pos4 v6)
    (pos5 v5)
    (pos6 v4)

)
(:goal (and 
    (pos1 v1)
    (pos2 v2)
    (pos3 v3)    
    (pos4 v4)    
    (pos5 v5)    
    (pos6 v6)    
    ))
;;;(:length (:serial 13) (:parallel 13))
)
\end{lstlisting}

In diesem Beispielproblem sind die Zahlen zun�chst wie folgt angeordnet: \(1,3,2,6,5,4\) und das Ziel ist es, die Zahlen in aufsteigender Reihenfolge zu ordnen. Wenn diese Dom�ne und dieses Problem durch den Online-Planer laufen, ergibt sich der folgende Plan:\\

\begin{lstlisting}
Plan:
    (rot1 v1 v3 v2 v6 v5 v4)
    (rot2 v1 v5 v6 v2 v3 v4)
    (rot1 v1 v5 v4 v3 v2 v6)
\end{lstlisting}

Wenn wir die einzelnen Schritte des Plans pr�fen, k�nnen wir feststellen, dass er das angestrebte Ziel tats�chlich erreicht:
\begin{itemize}
    \item \textbf{1. Schritt:} \(1,3,2,6,5,4 \rightarrow rot1 \rightarrow 1,5,6,2,3,4\)
    \item \textbf{2. Schritt:} \(1,5,6,2,3,4 \rightarrow rot2 \rightarrow 1,5,4,3,2,6\)
    \item \textbf{3. Schritt:} \(1,5,4,3,2,6\ \rightarrow rot1 \rightarrow 1,2,3,4,5,6\)
\end{itemize}