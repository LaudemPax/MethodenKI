\section{Plan-Domain-Definition-Language}

Wie beim Probleml�sen ist auch bei der Planung die Problemformulierung entscheidend. Um die Leistung von Planern zu vergleichen und Modellierungsarbeiten wiederzuverwenden, wird eine standardisierte Modellierungssprache f�r die Planung angestrebt. Hier kommt die Plan Domain Definition Language (PDDL) ins Spiel. 

Unter Verwendung von PDDL erfolgt die Modellierung, indem die Dom�ne in einer Datei, \textbf{domain.pddl}, und das Problem in einer anderen Datei, \textbf{problem.pddl}, beschrieben wird. \\

\begin{lstlisting}[caption={Beispiel domain.pddl},captionpos=b]
(define (domain <domain name>)
    (:types location locatable - object ... ;;; welche Objektklassen
    (:predicates (at ?obj - locatable ?loc - location) ;;; welche Pr�dikate
    (:action LOAD-TRUCK :parameters ... ;;; welche Aktionen
\end{lstlisting}

\begin{lstlisting}[caption={Beispiel problem.pddl},captionpos=b]
(define (problem DLOG-2-2-2)
    (:domain driverlog) ;;; Datei mit Domainenbeschreibung
    (:objects driver1 ? driver ... ;;; welche konkrete Objekte
    (:init (at driver1 s2) (at driver2 s2) ...     ;;; Startzustand
    (:goal (and (at driver1 s1) (at truck1 s1) ... ;;; Zielzustand
\end{lstlisting}

