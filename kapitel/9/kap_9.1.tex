\section{Einf�hrung in die Planung}

\paragraph{Plan vs Planen}

Wenn es um Handlungsplanung geht, ist es wichtig, den Unterschied zwischen einem Plan und einer Planung zu kennen. Ein \textbf{Plan} ist eine Struktur, die Darstellungen von Aktionen und Zielen enth�lt, die verwendet werden, um �ber die Wirkung zuk�nftiger Aktionen zu urteilen und die zielgerichtete Ausf�hrung von Aktionen zu beeinflussen. Das \textbf{Planen} ist die Generierung von Handlungsabl�ufen, die von einer gegebenen Situation ausgehen und zu einer gew�nschten Zielsituation f�hren.

\paragraph{Arten von Pl�nen}

Planungsaufgaben sind oft komplexer Natur und erfordern die Verarbeitung unterschiedlichster Informationen. Es gibt im Allgemeinen drei Arten von Pl�nen, lineare Pl�ne, nichtlineare Pl�ne und hierarchisch strukturierte Pl�ne. Ein \textbf{linearer Plan} ist eine Abfolge von Handlungen, die befolgt werden k�nnen, um das Ziel zu erreichen. Ein \textbf{nichtlinearer Plan} ist eine Menge von Aktionen, die nur teilweise nach Ausf�hrungszeit geordnet sind, da einige Aktionen parallel ausgef�hrt werden k�nnen. Ein \textbf{hierarchisch strukturierter Plan} ist ein Plan, der aus Teilpl�nen besteht.

\paragraph{Spezifikation einer Planungsaufgabe}

Zur Bearbeitung einer Planungsaufgabe mit einem Computer werden eine geeignete Sprache zur Definition der Aufgabe und ein Planungsalgorithmus zur L�sung der Aufgabe ben�tigt. Die Sprache muss in der Lage sein, die Dom�ne der Aufgabe, diesen Zustand des Suchraums und die erlaubten Aktionen zu definieren. Was den Algorithmus anbelangt, so bestehen die beiden Implementierungsmethoden darin, ein allgemeines Verfahren zu verwenden, z. B. eine Suche mit Backtracking, oder einen f�r die Aufgabe spezifischeren Prozess zu verwenden (der normalerweise effizienter ist).