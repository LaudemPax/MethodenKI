\section{Erlernen von Begriffen}

Der menschliche Geist ist in der Lage, Wahrnehmungen, Ideen und Vorstellungen zusammenzufassen und zu strukturieren, was den Austausch von Ideen durch Sprache erm�glicht. Mittels maschinellem Lernen wird versucht, diesen Vorgang von Computern zu simulieren.

Um diese Art des Lernens von Begriffen auszuf�hren, sind zwei Megnen von Objekten erforderlich, M+ und M-. Menge M+ enth�lt positive Beispiele daf�r, was in der gew�nschten Kategorie ist, und Menge M- enth�lt negative Beispiele (Objekte mit denselben Merkmalen wie in M+, aber nicht Teil der gew�nschten Kategorie).

Um das Modell zu trainieren, wird ein Prozess namens induktives Lernen verwendet. Bei diesem Verfahren beginnt man mit den Positivbeispielen, M+ und verallgemeinert eine Hypothese H, so dass die Eigenschaften der Positivbeispiele abgedeckt sind. Danach werden die Negativbeispiele herangezogen, sodass die Negativmerkmale nicht mehr von der Hypothese abgedeckt werden.

Das Ergebnis eines solchen Prozesses kann in Form von IF-, THEN-Regeln oder in Form einer Disjunktion von Regeln vorliegen, die die Eigenschaften eines Objekts darstellen k�nnen.