\section{Lernen auf der Grundlage von Daten}

Beim maschinellen Lernen sind \textbf{Daten} grunds�tzlich alle maschinenlesbaren digitalen Darstellungen von Informationen. Ein \textbf{Datensatz} ist eine Struktur in Form von \(d=[d1, ..., dn]\), die aus mehreren Daten erstellt wird. Eine Sammlung von Datens�tzen, die von maschinellen Lernprogrammen verarbeitet werden sollen, werden als \textbf{Datenkorpus} bezeichnet.

\subsection{Einf�hrung in Datens�tzen}

\paragraph{Format eines Datensatzes}

Das Format eines Datensatzes h�ngt von der Art der zu verarbeitenden Daten ab. F�r Datens�tze mit booleschen Typen, nominalen oder numerischen Typen oder Zeitstempeln sind die gebr�uchlichsten Datenformate Comma Seperated Values (CSV), Attribute-Relation-File-Format (ARFF), LIPSVM und JSON.

\paragraph{Arten von Merkmalen}

Bei der Arbeit mit Datens�tzen sind vor allem die im Datensatz gespeicherten Merkmale zu beachten. Es gibt verschiedene Arten von Merkmalen, und die Art beeinflusst die Art der Verarbeitung, die durchgef�hrt werden kann:

\begin{itemize}
    \item \textbf{Nominale Merkmale:} Merkmale, die f�r die Anwendung nicht in eine bedeutungstragende Orndnung gebracht werden konnten. Beispiele: Bool=\{True, False\}, Farbe=\{Rot, Gr�n, Blau\}, Geschlecht=\{w, m, d\}
    \item \textbf{Ordinale (geordnete) Merkmale:} Merkmale, die f�r die Anwendung in eine bedeutungstragende Orndnung gebracht werden k�nnen. Beispiele: Wochentage=\{Mon, Di,  Mi, \ldots\}, Noten=\{sehr gut, gut, \ldots\}
    \item \textbf{Numerische Merkmale:} Die Merkmale sind Zahlenwerte, die einen metrischen Abstand definieren k�nnen. Beispiele: Gr��e, Gewicht, Preis
\end{itemize}

\subsection{Bearbeiten mit Merkmalen}

\paragraph{Vergleich von bin�ren Merkmalen}

Gegeben seien zwei Objekte X, Y mit bin�ren Eigenschaften \(b_1, \ldots\, b_k\). Folgende Variablen werden verwendet:

\begin{itemize}
    \item \textbf{pp=} Anzahl positiver Merkmalen, die sich auf beide Objekte beziehen.
    \item \textbf{pn=} Anzahl von Merkmalen, die sich positiv auf X, aber negativ auf Y beziehen.
    \item \textbf{np=} Anzahl von Merkmalen, die sich negativ auf X, aber positiv auf Y beziehen.
    \item \textbf{nn=} Anzahl negative Merkmalen, die sich auf beide Objekte beziehen.
\end{itemize}

Um den Abstand zwischen zwei Objekten zu berechnen, die diese bin�ren Eigenschaften teilen, kann das Folgende verwendet werden:

\emph{Hamming Distanz:}

\[D_{Ham} (x,y) = pn + np\]

\emph{Simple Matching Distanz:}

\[D_{SMD} (x,y) = \frac{pn + np}{pp+pn+np+nn}\]

\emph{Jaccard Distanz:}

\[D_{JD} (x,y) = \frac{pn + np}{pp+pn+np}\]

\paragraph{Vergleich von nominalen Merkmalen}

Das Problem mit nominalen Merkmalen ist, dass es nicht m�glich ist, sie sinnvoll in eine Rangfolge zu ordnen. Um dies zu l�sen, kann die nominalen Merkmalen binarisiert werden, indem man f�r jeden m�glichen Merkmalswert ein bin�res Merkmal zuordnet.

Zum Beispiel f�r den Familienstand von zwei Personen, \(P_1, P_2\), k�nnen drei boolesche Attribute zugeordnet werden:-

\[ledig(P) -> \{TRUE, FALSE\}\]

\[verheiratet(P) -> \{TRUE, FALSE\}\]

\[geschieden(P) -> \{TRUE, FALSE\}\]

Und jetzt k�nnen die gleichen Methoden f�r bin�re Merkmale wiederverwendet werden.

\paragraph{Vergleich von ordinalen Merkmalen}

F�r ordinale Merkmale \(m \in \{o_1, o_2, \ldots, o_m\}\) ist der Ansatz einfacher. Es m�ssen lediglich die Rangordnungen der Elemente ermittelt werden und als Differenz der Rangordnungen kann die Distanz berechnet werden:

\[D_m(X_m, Y_m) = | Rang(X_m) - Rang(Y_m) |\]

\paragraph{Vergleich von metrischen Merkmalen}

F�r diesen ist der Ansatz �hnlich wie bei nominalen Merkmalen, da die Differenz zwischen ihnen berechnet wird:

\[D_m(X_m, Y_m) = | X_m - Y_m |\]

\subsection{Auswahl von Merkmalen}

Es ist wichtig, die richtigen Merkmale f�r die Aufgabe auszuw�hlen. Generell sollte man Merkmale w�hlen, die f�r alle beobachteten Objekte gelten, einfach und zuverl�ssig bestimmt werden k�nnen und f�r die jeweilige Aufgabe relevant sind.

F�r maschinelle Lernanwendungen ist es jedoch gut, so wenige Merkmale wie n�tig zu verwenden, um die Komplexit�t des Modells zu reduzieren. Anhand eines Korrelationsquotienten ist es m�glich, relevante Merkmalspaare in Datens�tzen zu finden.

Vor der Nutzung eines Datensatzes hilft es, eine sogenannte Datenbereinigung durchzuf�hren. Dazu sollte man Ausrei�er oder unplausible Werte eliminieren, die Daten neu strukturieren, um sie an maschinelle Lernalgorithmen anzupassen und fehlende Werte zu behandeln.