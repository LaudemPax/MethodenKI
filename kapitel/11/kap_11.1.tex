\section{Lernf�higen Systeme}

Ein lernf�higes System wertet aufbereitete Daten aus, um relevantes Wissen f�r die Durchf�hrung einer Aufgabe zu extrahieren. Dieses Wissen wird im maschinellen Lernen als ``Modelle'' bezeichnet und entweder deklarativ (symbolisch), numerisch (subsymbolisch) oder prozedural (Programmcode) dargestellt.

Das Lernziel ist, Aufgaben effizienter und mit weniger Fehlern zu l�sen, wenn das Modell verfeinert wird. Es kann auch erw�nscht sein, dass ein maschinelles Lernmodell neue Aufgaben �bernehmen kann, indem es neue Begriffe und ihre Bedeutungen, neue Assoziationen und neue Kausalzusammenh�nge lernt.

\subsection{�berwachtes vs Un�berwachtes Lernen}

Es gibt im Allgemeinen zwei Arten von maschinellem Lernen: �berwachtes Lernen und un�berwachtes Lernen. 

\textbf{Beim �berwachten Lernen} wird zwischen einer Lernphase und einer Testphase unterschieden. In der Testphase existiert ein ``Lehrer'', der den Beispieldatensatz klassifiziert und den Prozess �berwacht. W�hrend des Lernens trainiert sich das Modell anhand des gelabelten Datensatzes des Lehrers, und w�hrend der Testphase verwendet das Modell dann einen noch nicht zuvor gesehenen Datensatz ohne Labels und muss zeigen, wie gut es die Daten klassifizieren kann.

\textbf{Beim un�berwachten Lernen} lernt das Modell selbstst�ndig. Dies kann durch Data Mining erfolgen, indem ein Modell mit Daten ``gef�ttert'' und nach Mustern gesucht wird, oder indem positive oder negative ``Erfahrungen'' verwendet werden, um ein Modell zum Lernen zu bringen. Die Verwendung dieses Ansatzes mit Erfahrungen wird als best�rkendes Lernen bezeichnet, bei dem das Modell basierend auf den Ergebnissen seines Trainings positives oder negatives Feedback erh�lt. Das Modell �ndert sich dann basierend auf dem Feedback.

\subsection{Algorithmische Ans�tze zum maschinellen Lernen}

Es gibt einige Ans�tze f�r maschinelles Lernen, die h�ufig verwendet werden:

\begin{itemize}
    \item \textbf{Induktive Lernans�tze:} Anhand konkreter Beispiele wird der allgemeine Fall abgeschlossen. Ein gutes Beispiel ist das Lernen von Begriffen.
    \item \textbf{Statistiche Ans�tze:} Anhand konkreter Beispiele wird auf die H�ufigkeit des wahrscheinlichsten Falles geschlossen. Beispiele: Klassifikation mit ``ZeroR'', ``Naive-Bayes''.
    \item \textbf{Geometrische Ans�tze:} Betrachte Datens�tze als Punkte in einem mehrdimensionalen Koordinatenraum und finde Muster in den Positionen der Datenpunkte. Beispiele: Regression, Support-Vector-Machines.
    \item \textbf{Informationstheoretische Ans�tze:} Einen Datensatz so in Teilmengen aufteilen, dass sich die Teilmengen m�glichst stark voneinander unterscheiden. Beispiele: EM-Clustering, Entscheidungsbaum-Lerner.
    \item \textbf{Symbolisch vs  Subsymbolisch Ans�tze:} 
    \begin{itemize}
        \item Symbolisch: Das erlernte Wissen wird symbolisch dargestellt, z. B.: eine Menge von Regeln, pr�dikatenlogische Formeln, Entscheidungsb�ume etc.
        \item Subsymbolisch/Nueronal: Wissen wird nicht symbolisch gespeichert. z.B: Nueronale Netze, man lernt numerische Gewichte-
    \end{itemize}
    \item \textbf{Pipeline Ansatz (klasssich):} Der Lernprozess wird in verschiedene Schritte unterteilt und mit speziellen Techniken bearbeitet.
    \item \textbf{End-to-End learning/ black-box KI:} Der Lernprozess ist nicht in viele Schritte unterteilt. Stattdessen wird ein tiefes neurales Netz verwendet.
    \item \textbf{Transfer Learning:} Unter Verwendung von Daten aus Quelle A soll eine Schlussfolgerung von Dom�ne B gefunden. Dies geschieht, wenn zu wenige Datens�tze zum Trainieren f�r Dom�ne B vorhanden sind. Dies wird angewendet, indem ein geeigneter Indikator im Datensatz f�r A ausgew�hlt wird, der f�r Dom�ne B verwendet werden kann.
    \item \textbf{Lernen durch Vergleichen, Ausz�hlen und aufteilen von Datens�tzen:} Der Algorithmus durchl�uft den Datensatz und versucht mit einem deterministischen Programm, das Lernziel zu erreichen. Beispiele: Naive Bayes, Entscheidungsbaum-Lerner.
    \item \textbf{Iterative Lernverfahren/ Lernen in Epochen:} Der Lernprozess durchl�uft den Datensatz mehrmals, um das Modell iterativ zu verfeinern. Beispiele: Neuronale Netze, Q-Learning, Regressionsverfahren mit mehreren Variablen.
\end{itemize}