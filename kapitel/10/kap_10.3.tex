\section{Probabilistisches Schlussfolgern}

Unter Verwendung der Regeln der Wahrscheinlichkeitsrechnung ist es m�glich, Anwendungsmodelle zu erstellen, in denen verschiedene probabilistische Schlussfolgerungen unterscheidet werden.

Aus Fakten mit a priori bedingter Wahrscheinlichkeit \(P(B | A_i)\) kann auf die Wahrscheinlichkeit des Ereignisses B geschlossen werden. Aus dem Eintreten des Ereignisses B kann nach der Bayes-Formel die wahrscheinlichste Ursache von B bestimmt werden.

Bayes'sche Graphen werden in der Praxis verwendet, wobei kausale Beziehungen als bedingte Wahrscheinlichkeiten dargestellt werden. Zur Modellierung von Ereignissen werden sogenannte Zufallsvariablen und Wahrscheinlichkeitsverteilungen verwendet.

Zufallsvariablen sind Variablen die durch Zufallsprozesse zugewiesen werden. Es gibt drei Typen: boolesch (z.B M�nzwerf), diskret (z.B W�rfelwurf) und kontinuierlich (z.B Temperatur Morgen).

Eine Wahrscheinlichkeitsverteilung ist eine Zufallsvariable mit m�glichen Werten von 0 bis 1. Dieser Wert wird verwendet, um die Wahrscheinlichkeit darzustellen, dass ein Ereignis eintritt, wobei 1 eine Wahrscheinlichkeit von 100\% und 0 eine Wahrscheinlichkeit von 0\% darstellt.

\subsection{Wissensbasierte Systeme mit probabilistischen Schlussfolgerungen}