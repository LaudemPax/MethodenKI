\section{Einschub: Wahrscheinlichkeitsrechnung}

Sei A ein Ereignis, das eintreten oder nicht eintreten kann, lautet die Definition der Wahrscheinlichkeit \(P(A) \in [0, \dots, 1]\):

\[ P(A) = \frac{\textnormal{Anzahl der F�lle in denen A eintritt (g�nstige F�lle)}}{\textnormal{Gesamtzahl der m�glichen F�lle, in denen A eintreten kann}} \]

Es gibt zwei Arten von Wahrscheinlichkeiten, \textbf{a priori Wahrscheinlichkeit} und \textbf{bedingte Wahrscheinlichkeit}. A priori Wahrscheinlichkeit, \(P(A)\) dr�ckt aus, wie wahrscheinlich das Ereignis A eintreten wird. Bedingte Wahrscheinlichkeit, \(P(B | A)\), dr�ckt aus, wie wahrscheinlich das Ereignis B in Abh�ngigkeit von der Wahrscheinlichkeit des Ereignisses A eintritt.

\paragraph{Bedingte Wahrscheinlichkeit}

Definition bedingte Wahrscheinlichkeit:

\[P(B|A) = \frac{P(A \cap B)}{P(A)}\]

sowie:

\[P(A|B) = \frac{P(A \cap B)}{P(B)}\]

Bei der Berechnung von Wahrscheinlichkeiten ist zu unterscheiden zwischen dem Fall, dass ein Ereignis von einem anderen abh�ngig ist, und dem Fall, dass die beiden Ereignisse unabh�ngig voneinander sind. 

Die Wahrscheinlichkeit, zwei gleichfarbige Kugeln aus einer Urne mit gemischten Kugeln auszuw�hlen, ist ein klassisches Beispiel f�r einen Fall, in dem das Ereignis A, das die erste Kugel ausw�hlt, sich direkt auf die Wahrscheinlichkeit von B auswirkt, eine zweite Kugel der gleichen Farbe auszuw�hlen. Andererseits ist die Wahrscheinlichkeit, zwei gleiche Augenzahlen zu w�rfeln, unabh�ngig vom ersten Wurf.

\paragraph{Totale Wahrscheinlichkeit}

Zerlegt man den Ereignisraum G in die paarweise disjunkte Mengen (disjunkt = sich gegenseitig ausschlie�ende Ereignisse) \(A_1\) bis \(A_n\), und gilt f�r jedes \(A_i\) dass \(P(A_i) > 0\), so l�sst sich f�r ein belibiges Ereignis B die Wahrscheinlichkeit \(P(B)\) berechnen durch:

\[P(B) = \sum_{i=1}^{n} P(B \cap A_i) = \sum_{i=1}^{n} P(A_i) * P(B | A)\]

\paragraph{Bayes'sche Formel}

Seien \(A_1\) bis \(A_n\) eine disjunkte Zerlegung des Ereignisraums \(G\) und sei \(B\) mit \(P(B) \neq 0 \) ein bereits eingetretenes Ereignis. Dann kann man ein \(P(A_i | B)\) wie folgt berechnen:

\[ P(A_i | B) = \frac{P(A_i) * P(B | A_i)}{\sum P(A_k) * P(B | A_k)} \textnormal{ f�r } 1 \le i \le n \textnormal{ und } 1 \le k \le n\]