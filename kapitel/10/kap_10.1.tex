\section{Einf�hrung in unsicheres Wissen}

Die meisten �u�erungen unsicheren Wissens sind auf unvollst�ndiges Wissen oder Zuf�lligkeit zur�ckzuf�hren. Dieses unvollst�ndige Wissen hat paar m�gliceh Quellen: Ein Ausdruck kann sich auf ein zuk�nftiges Ereignis beziehen, aufgrund von Leistungseinschr�nkungen auf Sch�tzungen beruhen oder aus einer unvollst�ndigen Modellierung eines Kontexts in einer Dom�ne gefolgert werden, was alles zu Unsicherheit f�hren kann.

Unsicherheit l�sst sich nicht mit klassischen Logik (wie Aussagenlogik oder PL1, etc) oder mit klassischen Wenn-dann-Regeln formulieren. Dies wirft nat�rlich die Frage auf, wie wir Unsicherheit beschreiben und f�r Schlussfolgerungen und Inferenzen nutzen k�nnen?

\paragraph{Modellierung von unsicherem Wissen}

Die Modellierung unsicheren Wissens kann auf verschiedene Arten erfolgen. Ein Ansatz besteht darin, Ausdr�cken Wahrscheinlichkeiten zuzuweisen und diese Wahrscheinlichkeiten zu verwenden, wenn Schlussfolgerungen gezogen werden. Ein Ansatz besteht darin, Ausdr�cken Wahrscheinlichkeiten zuzuweisen und diese \textbf{Wahrscheinlichkeiten} zu verwenden, wenn Schlussfolgerungen gezogen werden. Ein anderer Ansatz w�re die Zuweisung von \textbf{``Sicherheitswerte''}, um auszudr�cken, wie sicher man sich einer bestimmten Fakt oder Regel ist. Schlie�lich kann Unsicherheit auch unter Verwendung \textbf{nicht monotoner Schlie�en} und \textbf{Default-Logik} modelliert werden, was bedeutet, bestimmte ``Defaults'' anzunehmen, die sp�ter widerrufen werden k�nnen.