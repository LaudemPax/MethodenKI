\section{Ausdrucksst�rke Logiksprachen}

Die Aussagenlogik funktioniert in einigen F�llen, ist aber auf die Modellierung von Fakten beschr�nkt. Dies kann zu kompliziert werden, wenn alle strukturellen Beziehungen explizit als Fakten dargestellt werden m�ssen. Mit Aussagenlogik gibt es keine Quantifizierung, keine Relationen und keine Funktionen. Logikprache muss also erweitert werden, um aussagekr�ftiger zu sein.

\subsection{Pr�dikatenlogik 1. Stufe}

Die logische Sprache wird um Pr�dikate (boole'schen Funktionen), Variablen und Quantoren (erm�glicht es, �ber eine Menge von Objekten mit Hilfe einer Variablen zu sprechen) erweitert.

Pr�dikatenlogik 1. Stufe (First Order Logic) beschreibt Objekte, ihre Eigenschaften und ihre Beziehungen zueinander. Es gibt bei Pr�dikatenlogik auch eine Unterscheidung zwischen Syntax und Semantik.

\paragraph{Syntax: Lexikalischer Teil}

\begin{itemize}
    \item Konstanten-Symbole: TRUE, FALSE, A, B, Hans, Paul\dots
    \item Variablen-Symbole: x, y, \dots
    \item Funktions-Symbole: plus, minus, mul, Vater\_von\dots
    \item Pr�dikaten-Symbole: student, hat\_Computer, P, Q\dots
    \item Logische Verkn�pfungen: \(\vee \wedge \neg \Leftrightarrow \Rightarrow \)
    \item Quantoren: \(\exists , \forall\)
    \item Gleicheit: \(=\)
\end{itemize}

\paragraph{Syntax: Struktureller Teil}

\begin{itemize}
    \item \textbf{Terme:} Konstaten-Symbole, \( f(t_1, t_2, t_3, \dots, t_n) \) wenn \(f\) eine Funktionsymbol und \(t_1\) bis \(t_n\) Terme sind.
    \item \textbf{Atomare Formel:} \(P(t_1, t_2, t_3, \dots, t_n)\) wenn Pr�dikatensymbol \(P\) und \(t_1\) bis \(t_n\) Terme sind.
    \item \textbf{Komplexe Formel:} jede atomare Formel ist eine Formel. Wenn R und S Formeln sind und x ein Variablen-Symbol dann sind folgende Formeln: \(R\), \(\neg R\), \(R \wedge S\), \(R \vee S\). \(R \Rightarrow S\), \(\exists x: R\), \(\forall x:R\)
\end{itemize}

\paragraph{Beispiele von Sachverhalten-Formulierung in Pr�dikatenlogik}

\begin{itemize}
    \item Alle Kinder lieben Eiscreme: \(\forall x:Kind(x) \Rightarrow liebt\_Eiscreme(x)\)
    \item Es gibt einen Baum der Nadeln hat: \(\exists x:Baum(x) \wedge hat\_Nadeln(x)\)
    \item Die Mutter einer Person ist dessen weibliches Elternteil: 
    \[\forall x \forall y : Mutter(x,y) \Leftrightarrow (weiblich(x) \wedge Elternteil(x,y))\]
\end{itemize}

\subsection{Semantik der Pr�dikatenlogik}

Es wird eine Interpretationsfunktion verwendet, die der Aussagelogik �hnelt, aber komplexer ist. Der Wahrheitsgehalt einer Formel wird durch die Struktur, \(S = (U,I)\) bestimmt, wobei U das ``Universum'' (beobachteter Gegenstandsbereich) und I die Interpretationsabbildung (die Bestandteile einer Formel, die die Objekte und Relationen im Gegenstandsbereich abbildet) sind.