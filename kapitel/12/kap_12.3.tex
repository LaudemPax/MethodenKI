\section{Klassifikation mit Entscheidungsb�umen}

\subsection{Erlernen einer Entscheidungs-Regel}
\label{section:oneR}
Gegeben sind \(m\) n-dimensionalte Datens�tze \(d_i = [a_1, a_2, \ldots, a_n]\) einer relationalen Datenbank, die interpretiert werden als Argument-Wertepaare einer n-1 stelligen, 1-wertigen Funktion \(f(a_1, a_2, \ldots, a_{n-1})\). Es wird ein einfacher Klassifikator der Form \textbf{IF} \(a_r = w\) \textbf{THEN} \(a_n = k\) gesucht.

Ein Ansatz daf�r w�re, aus dem Trainingsdatensatz ein Attribut \(a_r\) zu bestimmen, das die beste Vorhersage f�r das Zielattribut \(a_n\) liefert, wenn das Attribut \(a_r\) den Wert \(k\) hat.

\subsection{Erlernen einer Entscheidungs-Baums}

Gegeben sind \(m\) n-dimensionalte Datens�tze \(d_i = [a_1, a_2, \ldots, a_n]\) einer relationalen Datenbank, die interpretiert werden als Argument-Wertepaare einer n-1 stelligen, 1-wertigen Funktion \(f(a_1, a_2, \ldots, a_{n-1})\). Es wird ein aber diesmal eine baumartige Entscheidungsstruktur, der einem Datensatz \(d=[a_1, a_2, \ldots, a_{n-1}]\) der passende Wert des Zielattributs \(a_n\) zuordnet.

Der aus den Beispieldaten erstellte Entscheidungsbaum sollte die Attribute so anordnen, dass: 

\begin{itemize}
    \item alle Beispieldatens�tze korrekt klassifiziert werden k�nnen
    \item die Anzahl der Entscheidungen auf ein Minimum beschr�nkt werden
    \item neue Datens�tze klassifiziert werden k�nnen indem diesen Entscheidungsbaum verwendet wird.
\end{itemize}

Zwei m�gliche Strategien zum Aufbau eines Entscheidungsbaums sind:

\begin{itemize}
    \item \textbf{Strategie 1:} Von der Wurzel ausgehend w�hle stets dasjenige Attribut \(a_i\) als n�chstes Auswahlkriterium, welches von den verbleibenden Beispieldatens�tzen die wenigsten ``fehlklassifiziert''. Das bedeutet, dass der zuvor in Kapitel \ref{section:oneR} beschriebene Ansatz rekursiv angewendet wird, um die Attribute f�r den Entscheidungsbaum auszuw�hlen.
    \item \textbf{Strategie 2:} Aus der Wurzel des Baums wird das Attribut ausgew�hlt, das den gr��ten Informationsgewinn hat. Bei dieser Methode wird eine Gleichung zur Berechnung des Informationsgewinns jedes Attributs (Informationstheorie Shanon 1948) rekursiv verwendet, um die Attribute und die Werte des Entscheidungsbaums auszuw�hlen.
\end{itemize}