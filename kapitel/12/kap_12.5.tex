\section{Vertiefung: Klassifikatoren Trainieren mit Python}

Um den gesamten Prozess des maschinellen Lernens besser zu verstehen, wurde ein kleines Projekt zum Trainieren eines Klassifikators durchgef�hrt. Dieses Beispiel basiert auf einem Online-Artikel von Medium\cite{classification-online}. In diesem Beispielprojekt werden die �blichen Schritte zur Durchf�hrung von maschinellem Lernen befolgt:
\begin{enumerate}
    \item Data-Preprocessing.
    \item Aufteilen des Datensatzes in Trainings- und Testdatens�tze.
    \item Trainieren des Modells Trainieren der Modelle (eines mit einem Kneighbors-Algorithmus und eines mit einem Entscheidungsbaum).
    \item Klassifikatoren auswerten
\end{enumerate}

In dieser Trainingsaufgabe wird das Modell mit einem Bankdatensatz von Kaggle\cite{kaggle-dataset} darauf trainiert, abh�ngig von einigen Attributen vorherzusagen, ob jemand eine Einzahlung t�tigen wird oder nicht.

\subsection{Data Pre-Processing}

Im Vorverarbeitungsschritt entf�llt die Spalte ``duration'', da diese laut Dokumentation erst verf�gbar ist, nachdem die Label-Spalte bekannt ist. Danach werden die Daten skaliert, um zu vermeiden, dass Ausrei�erdaten das Modell erheblich beeintr�chtigen, und schlie�lich werden die Daten zur einfacheren Verarbeitung mit einer One-Hot-Codierung codiert.

\begin{lstlisting}[language=python]
# load dataset
df_bank = pandas.read_csv('dataset/bank.csv')

# drop duration column
df_bank = df_bank.drop('duration', axis=1)

# create a copy
df_bank_ready = df_bank.copy()

# scaling numberic data to avoid significant outlier presence
scaler = StandardScaler()
num_cols = ['age', 'balance', 'day', 'campaign', 'pdays', 'previous']
df_bank_ready[num_cols] = scaler.fit_transform(df_bank[num_cols])

# Encoding categorical data
encoder = OneHotEncoder(sparse=False)
cat_cols = ['job', 'marital', 'education', 'default', 'housing', 'loan', 'contact', 'month', 'poutcome']

# Encode Categorical Data
df_encoded = pandas.DataFrame(encoder.fit_transform(df_bank_ready[cat_cols]))
df_encoded.columns = encoder.get_feature_names(cat_cols)

# Replace Categotical Data with Encoded Data
df_bank_ready = df_bank_ready.drop(cat_cols ,axis=1)
df_bank_ready = pandas.concat([df_encoded, df_bank_ready], axis=1)

# Encode target value
df_bank_ready['deposit'] = df_bank_ready['deposit'].apply(lambda x: 1 if x == 'yes' else 0)
\end{lstlisting}

\subsection{Splitting the datasets}

Die Daten werden in Trainings- und Testdatens�tze aufgeteilt, wobei 80\% der Daten f�r das Training und 20\% f�r Tests verwendet werden.

\begin{lstlisting}[language=python]
# Select Features
feature = df_bank_ready.drop('deposit', axis=1)

# Select Target
target = df_bank_ready['deposit']

# Set Training and Testing Data
from sklearn.model_selection import train_test_split
X_train, X_test, y_train, y_test = train_test_split(feature , target, 
                                                    shuffle = True, 
                                                    test_size=0.2, 
                                                    random_state=1)

\end{lstlisting}

\subsection{Modelling and evaluation}

Anschlie�end erfolgte die Modellierung der beiden Klassifikatortypen und die Auswertung erfolgte anhand einer Reihe von Metriken sowie einer Konfusionsmatrix.

\begin{lstlisting}[language=python]
# Evaluation function
def evaluate_model(model, x_test, y_test):
    from sklearn import metrics

    # Predict Test Data 
    y_pred = model.predict(x_test)

    # Calculate accuracy, precision, recall, f1-score, and kappa score
    acc = metrics.accuracy_score(y_test, y_pred)
    prec = metrics.precision_score(y_test, y_pred)
    rec = metrics.recall_score(y_test, y_pred)
    f1 = metrics.f1_score(y_test, y_pred)
    kappa = metrics.cohen_kappa_score(y_test, y_pred)

    # Calculate area under curve (AUC)
    y_pred_proba = model.predict_proba(x_test)[::,1]
    fpr, tpr, _ = metrics.roc_curve(y_test, y_pred_proba)
    auc = metrics.roc_auc_score(y_test, y_pred_proba)

    # Display confussion matrix
    cm = metrics.confusion_matrix(y_test, y_pred)

    return {'acc': acc, 'prec': prec, 'rec': rec, 'f1': f1, 'kappa':kappa, 'fpr': fpr, 'tpr': tpr, 'auc': auc, 'cm': cm}

# Model training: decision tree
decision_tree_model = tree.DecisionTreeClassifier(random_state=0)
decision_tree_model.fit(x_train, y_train)

# Model training Kneighbors
kneighbors_model = KNeighborsClassifier()
kneighbors_model.fit(x_train, y_train)
\end{lstlisting}

\subsection{Ausf�hren des Projekts}

Das Projekt (vertiefungen-projekte/python) kann ausgef�hrt werden, indem zuerst die erforderlichen Bibliotheken installiert werden:

\begin{lstlisting}[language=bash]
pip install -r requirements.txt 
\end{lstlisting}

Und danach l�uft es einfach mit Python:

\begin{lstlisting}[language=bash]
classifier-training.py
\end{lstlisting}

Das Ausf�hren erzeugte die folgende Ausgabe:

\begin{lstlisting}
=====================================================
Decision tree classifier
=====================================================
Accuracy: 0.6336766681594268
Precision: 0.6215953307392996
Recall: 0.598314606741573
F1 Score: 0.6097328244274809
Cohens Kappa Score: 0.2648219403033133
Area Under Curve: 0.6322045136712157
Confusion Matrix:
    [[776 389]
    [429 639]]
=====================================================
K nearest neighbors classifier
=====================================================
Accuracy: 0.6869682042095835
Precision: 0.6981740064446831
Recall: 0.6086142322097379
F1 Score: 0.6503251625812906
Cohens Kappa Score: 0.3693851405429406
Area Under Curve: 0.7323909758724342
Confusion Matrix:
    [[884 281]
    [418 650]]
\end{lstlisting}