\section{Evaluation von Klassifikationen}

Beim maschinellen Lernen ist die Bewertung ein wichtiger Schritt f�r die Auswahl eines Klassifizierungsalgorithmus. Diese Auswertung erfolgt anhand von Testdaten, die nicht f�r das Training verwendet wurden. Ein guter Indikator f�r die G�te eines Klassifikators ist die Anzahl der Testdatens�tze, die der Klassifikator richtig und falsch klassifiziert hat. Dies l�sst sich am besten durch eine Konfusionsmatrix veranschaulichen.

\subsection{Evaluation mittels Konfusionsmatrix}

Bei einer Klassifikationsaufgabe, bei der Datens�tze von \(n\) Klassen \(K_1, \ldots, K_n\) klassifiziert werden, wird folgendes durchgef�hrt:
\begin{enumerate}
    \item Unter Verwendung der Klassen \(K_1, \ldots, K_n\) wird eine Matrix \(M\) konstruiert.
    \item In den diagonalen Feldern (\(K_i, K_i\)) von M wird die Anzahl der korrekt klassifizierten Datens�tze eingetragen.
    \item In den Feldern (\(K_r, K_s\)) wobei \(r \neq s\) wird die Anzahl der falsch klassifizierten Datens�tze eingetragen.
\end{enumerate}