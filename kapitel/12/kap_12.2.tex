\section{Instanzen-basiertes Klassifizieren}

Gegeben seien eine Menge von Beispieldatens�tzen \(d_1 \ldots d_n\), eine Menge von Klassen \(C_1 \ldots C_m\), in die die Datens�tze \(d_1 \ldots d_n\) unterteilt sind, und ein Datensatz \(d_{new}\). Es wird eine passende Klasse \(C_i\) f�r den neuen Datensatz \(d_{new}\) gesucht.

F�r diesen Fall kann ein Ansatz verwendet werden, der als ``k-n�chste-Nachbarn-Verfahren'' bekannt ist. Unter Verwendung dieses Prozesses wird die Anzahl k der n�chsten Nachbarn des neuen Datensatzes verwendet, um zu entscheiden, zu welcher Klasse der neue Datensatz geh�rt.

\paragraph{Ablauf des k-n�chste-Nachbarn-Verfahrens}

\begin{enumerate}
    \item Die Distanzen zwischen dem neuen Datensatz und den Beispieldatens�tzen werden berechnet.
    \item Die k n�chsten Datenpunkte und ihre Klassen werden bestimmt.
    \item Der neue Datensatz wird basierend auf der Mehrheitsklasse der k n�chsten Nachbarn entschieden.
\end{enumerate}