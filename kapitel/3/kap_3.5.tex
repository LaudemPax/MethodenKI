\section{KI-Suchverfahren}

Es gibt zwei Klassen von Suchverfahren: \textbf{blinde Suchverfahren} und \textbf{KI-Suchverfahren}. Blinde Suchverfahren sind auf einem bestimmten Schema basiert, das unabh�ngig von dem jeweiligen Problem ist. Einige Beispiele hierf�r sind die in den vorangegangenen Kapiteln behandelten Verfahren wie Breitensuche, Tiefensuche, Biridketionale Suche usw. 

KI-Suchverfahren hingegen nutzen problemspezifisches Vorwissen zur Eingrenzung des Suchraums. Es handelt sich um informierte heuristische Suchverfarhen.

\textbf{Blinde Suchverfahren} erfordern, dass eine L�sung durch systematische und ersch�pfende Suche in einem Suchgraphen gefunden wird, was ineffizient und kein problemspezifisches Wissen nutzt.

\textbf{Informierte Suchprozesse} hingegen nutzen problemspezifische Eigenschaften, um die Effizienz der Knotenexpansion zu verbessern.

\subsection{Greedy Search}
\label{section:greedy-search}
Die Greedy-Suche ist eine modifizierte Breitensuche (siehe Abschnitt \ref{section:breitensuche}), bei der nur die Knoten mit den geringsten Kosten in die Warteschlange aufgenommen werden. 

\paragraph{Ablauf von Greedy Search}

Wenn die Kosten des aktuellen Knotens zum Zielknoten unbekannt sind, \textbf{m�ssen diese Kosten gesch�tzt werden}, und dann wird der Nachbar mit den geringsten Kosten ausgew�hlt. Die Funktion, die diese Kosten sch�tzt, wird \textbf{heuristische Funktion} genannt.

Der Unterschied zwischen Greedy Search und Uniform Cost Search (siehe Abschnitt \ref{section:uniform-cost-search}) besteht darin, dass bei der Greedy Search \textbf{die Kosten von einem Knoten zum Zielknoten} berechnet werden und nicht von einem Knoten zum anderen.

\paragraph{Eigenschaften von Greedy Search}

Greedy Search bietet tendenziell schnelle L�sungen, die oft, aber nicht immer, der optimale Weg sind. 

Greedy Search ist �hnlich wie die Tiefensuche mit Backtracking, nicht vollst�ndig, und erfordert eine gute Heuristik f�r eine bessere G�te des Verfahrens.

\subsection{Der A* Algorithmus}

Der A*-Algorithmus ist ein neuer Ansatz, der auf Greedy Search (Abbschnitt~\ref{section:greedy-search}) und dem Dijkstra-Algorithmus~(Abbschnitt~\ref{section:dijkstra}) aufbaut. A* arbeitet basierend auf der Funktion:

\[f(n) = g(n) + h(n)\]

Wobie \(f(n)\) die gesch�tzten Kosten der billigsten L�sung ist, \(g(n)\) die Kosten f�r die Bewegung von der Ausgangszelle zur aktuellen Zelle, und \(h(n)\) die gesch�tzten Kosten f�r die Bewegung von der aktuellen Zelle zur Zielzelle. Mit der Funktion \(f(n)\) werden die Kosten berechnet, und der Rest des Algorithmus l�uft wie bei Greedy Serach ab.

\subsection{Pfadplannung mit A*}

\subsection{Pfadplannung in Computerspielen}

