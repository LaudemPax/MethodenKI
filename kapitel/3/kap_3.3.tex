\section{Wegsuche als systemstisches Ablaufen von Graphen}

Bei der Wegfindung mit Hilfe eines Graphen m�ssen ein \textbf{Startknoten} und eine \textbf{Funktion zum Testen}, ob der Zielknoten erreicht wurde, definiert werden. Mit Hilfe des Startknotens und dieser Funktion kann eine Folge von Knoten gefunden werden, die die Testfunktion erf�llen k�nnen. Wichtig ist, dass die f�r die Sequenz ausgew�hlten Knoten \textbf{benachbarte Knoten sind, die durch Kanten verbunden} sind.

\subsection{Generelle Wegfindungsstrategie}

Zyklen und Mehrfachbesuche sind zu vermeiden, da sie den Suchbaum exponentiell wachsen lassen k�nnen. Generell sollte man keinen bereits benutzten Weg nochmal gehen, keine Wege mit Zyklen kreieren, einen bereits besuchten oder ausgebauten Zustand nicht besuchen oder erzeugen.

\subsection{Generelle Bewertung von Suchverfahren}

Bewertungskriterien eines Pfadfindungsprozesses sind wie folgt:

\paragraph{Korrektheit}

Es ist wichtig, dass die Wegfindungsl�sung tats�chlich eine L�sung des Problems ist.

\paragraph{Vollst�ndigkeit}

Existiert eine L�sung, terminiert der Algorithmus nach endlicher Zeit und generiert eine L�sung.

\paragraph{Optimalit�t}

Die optimalste L�sung wird gefunden, wenn mehrere m�glich sind.

\paragraph{Zeitkomplexit�t}

Die Zeit, die im worst-case/average-case ben�tigt wird, um eine optimale L�sung zu finden.

\paragraph{Speicherkomplexit�t}

Wie viel Speicher, die im worst-case/average-case ben�tigt wird.

