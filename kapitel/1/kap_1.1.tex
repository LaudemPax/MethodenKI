\section{Was ist k�nstliche Intelligenz?}

Es gibt viele Beispiele daf�r, dass KI in der realen Welt aktiv ist, insbesondere mit der heutigen Technologie. Von relativ einfachen Anwendungen wie Computergegnern in rundenbasierten Spielen, �ber n�tzliche Expertensysteme bis hin zu selbstfahrenden Autos und KI-generierter Kunst. 

Doch auch heute noch ist es f�r Forscher und Experten schwierig, sich auf eine gute Definition von KI zu einigen. Versuche, KI zu definieren, scheinen entweder irrelevant zu werden, wenn sich die Technologie weiterentwickelt und als trivial angesehen wird, oder sie decken nicht das breite Spektrum der Anwendungen ab, f�r die KI eingesetzt wird.

Im Allgemeinen lassen sich die Anwendungen der KI in die folgenden drei Arten unterteilen:
\begin{itemize}
    \item \textbf{Artificial Narrow Intelligence}: Eine KI, die sich auf eine bestimmte Aufgabe spezialisiert hat und diese Aufgabe besser erledigt als ein Mensch. Diese KI ist aber nicht auf andere Aufgaben anwendbar, f�r die sie nicht definiert ist.
    \item \textbf{Artificial General Intelligence}: Eine KI, die in der Lage ist, die Welt zu beobachten und aus diesen Beobachtungen Schlussfolgerungen zu ziehen, wie es ein Mensch tun w�rde. Es besteht Einigkeit dar�ber, dass diese Stufe der KI noch nicht erreicht ist, aber man ist sich nicht einig, wann wir sie erreichen werden oder ob sie �berhaupt m�glich ist.  
    \item \textbf{Artificial Super Intelligence}: Eine KI, die dem Menschen in jeder Hinsicht �berlegen ist. Derzeit Science-Fiction, wirft aber ethische Fragen auf.
\end{itemize}

Das Problem bei der Definition von k�nstlicher Intelligenz beginnt wahrscheinlich damit, dass es keine einheitliche Definition dessen gibt, was ``nat�rliche Intelligenz'' ist. Dies wird in philosophischen und psychologischen Kreisen heftig diskutiert.

Zusammenfassend l�sst sich sagen, dass es keine einheitliche Definition von KI gibt und dass die einzigen n�tzlichen Anwendungen von KI, die wir heute haben, als die grundlegendste eingestuft werden, n�mlich als ``Artificial Narrow Intelligence''.