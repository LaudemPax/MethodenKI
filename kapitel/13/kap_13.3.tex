\section{Finden von Assoziationsregeln}

Bei einer bestimmten Anzahl von n-dimensionalen Datens�tzen in einer relationalen Datenbank werden die Zuordnungen von Attributen in Form von WENN-DANN-Regeln gesucht. Am besten l�sst sich dies anhand eines konkreten Beispiels erl�utern: der Warenkorbanalyse

\paragraph{Warenkorbanalyse}

Es stehen viele Produkte \(a_1, a_2, \ldots, a_n\) zur Auswahl. In einem Warenkorb ist jedes Produkt entweder vorhanden oder nicht. Eine \(m\) Anzahl von Warenk�rben wird untersucht, um Assoziationsregeln in Form von:

\[\textnormal{Wenn \(a_i\) im Korb dann auch \(a_k\)}\]

Jeder der Warenk�rbe wird als Datensatz (\(a_1, a_2, \ldots, a_n\)) dargestellt, wobei \(a_i\) der Platzhalter f�r ein Produkt ist. Alle m�glichen Produktkombinationen werden gebildet und die H�ufigkeit dieser Kombinationen in den Datens�tzen gez�hlt.

\paragraph{Bewertungskriterien f�r Regeln}

Es gibt einige Bewertungskriterien, anhand derer festgestellt werden kann, ob eine Regel relevant ist:

\begin{itemize}
    \item \textbf{Support:} bestimme den Verh�ltnis zwischen der H�ufigkeit des Auftretens eines Produkts in einer Kombination und der Anzahl der beobachteten Warenk�rbe. \[support(a_i) = \frac{\#a_i}{m}\] \[support(a_i \rightarrow a_k) = \frac{\#(a_i, a_k)}{m}\] wobei \(\#a_i\) ist die Anzahl der Vorkommen von Produkt \(a_i\) in allen Datens�tzen und \(\#(a_i, a_k)\) ist die Anzahl der Vorkommen der Kombination \((a_i, a_k)\) in allen Datens�tzen
    \item \textbf{Confidence}: Verh�ltnis des Auftretens der Kombination \((a_i ,a_k )\) zur Anzahl des Auftretens des Produkts \(a_i\) \[confidence(a_i \rightarrow a_k) = \frac{\#(a_i, a_k)}{\#a_i}\]
    \item \textbf{Expected Confidence:} Verh�ltnis des Auftretens der Regelkonklusion \(ak\) zur Anzahl aller betracteten Warenk�rbe. \[expected \textnormal{ } confidence(a_k) = \frac{\#a_k}{m}\]
    \item \textbf{Lift:} Verh�ltnis der Confidence der Regel zum Support der Regel der Regel-Konklusion \[lift(a_i \rightarrow a_k) = \frac{confidence(a_i \rightarrow a_k)}{expected\_confidence(a_k)}\]
\end{itemize}