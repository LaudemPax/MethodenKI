\section{Clusteranalyse}

\subsection{Klassifizieren vs Clusteranalyse}

``Clusteranalyse'' und ``Klassifizierung'' m�gen auf den ersten Blick �hnlich erscheinen, aber es gibt einen wesentlichen Unterschied: Bei der Klassifizierung sind die Klassen vordefiniert und den Datens�tzen Klassen zugeordnet, w�hrend beim Clustering der Datensatz nach �hnlichen Merkmalen gruppiert wird, und es ist nicht erforderlich, dass Klassen vordefiniert wurden.

Um ein Clustering zu erreichen, werden �hnliche Objekte zusammen gruppiert, w�hrend un�hnliche Objekte voneinander getrennt werden. Gesucht sind also \(k\) Klassen \(K_1 \ldots K_k\), die alle Objekte richtig gruppieren.

\subsection{Zielsetzung beim Clustering}

\begin{itemize}
    \item In einem Datensatz wird nach der Mindestanzahl von Clustern gesucht.
    \item Objekte im selben Cluster m�ssen so �hnlich wie m�glich sein.
    \item Objekte im selben Cluster m�ssen so �hnlich wie m�glich sein.
    \item Eventuell sollten die Cluster auch hierarchisch geordnet werden k�nnen
\end{itemize}