\section{Ziele beim Data Mining}

Data Mining ist der Prozess des Extrahierens und Entdeckens von Mustern in einer Datensammlung (Datenbank mit Datens�tzen). Es gibt einige Ans�tze, die f�r das Data Mining angewendet werden k�nnen. 

Bei der \textbf{Clusteranalyse} wird eine Menge von Objekten so aufgeteilt, dass Objekte in derselben Gruppe (Cluster genannt) einander �hnlicher sind als denen in anderen Gruppen (Clustern).

Abgesehen davon ist es auch m�glich, in einem Datensatz nach \textbf{Assoziationsregeln zu finden}. Beispielsweise kann es wertvoll sein zu wissen, welche Merkmalskombinationen h�ufiger vorkommen.

Dar�ber hinaus k�nnen Assoziationsregeln durch \textbf{Frequent Pattern Mining} verallgemeinert werden.

Abschlie�end werden mittels \textbf{Faktorenanalyse} korrelierte Merkmale von Datens�tzen analysiert und versucht, diese Merkmale zu ``Faktoren'' zu gruppieren.