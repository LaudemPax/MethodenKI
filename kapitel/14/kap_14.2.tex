\section{Der Q-Learning Ansatz}

Q-Learning ist ein Ansatz, um die optimale Policy \(\pi\) zu finden. Dies wird iterativ durchgef�hrt, um die Funktion \(Q^\pi(s,a) \rightarrow w \in IR\) zu approximieren. In jedem Iterationsschritt wird ein Zustand-Aktions-Paar in einer Tabelle gem�� der folgenden Q-Update-Regel aktualisiert:

\[Q(s_{i_{new}}, a_{j_{new}}) = Q(s_{i_{old}}, a_{j_{old}}) + \alpha * (r(s_i, a_j) + \gamma * max_k\{Q(s_{i+k}, a_k)\} - Q(s_{i_{old}}, a_{j_{old}}))\]

wobei \(\alpha\) die Lernrate, \(r\) die Belohnung im Zustand \(s_i\), wenn Aktion \(a_i\) ausgef�hrt wird, \(\gamma\) der ``Discount''-parameter zur Gewichtung des Look Ahead und \(Q\) die Look-Ahead-Funktion f�r zuk�nftige Belohnungen sind.

Die Aktionszustandspaare werden in einer Tabelle gespeichert, die als \textbf{Q-Tabelle} bekannt ist, die meist als geschachteltes Array implementiert ist. Die Q-Tabelle wird w�hrend des Lernprozesses iterativ aktualisiert, um eine Tabelle mit aussagekr�ftigen Werten f�r jede Aktion basierend auf dem aktuellen Zustand zu erhalten.

Q-Learning erfordert viel Aufwand, aber es hat sich gezeigt, dass, wenn die Anzahl der Iterationen gegen unendlich wird, die durch Q-Learning gefundene Richtlinie zur optimalen Strategie f�r die gegebenen Zustandsaktionspaare konvergiert.

In der Praxis ist es jedoch nicht m�glich, ewig zu trainieren, daher sind Abbruchkriterien erforderlich: Abbruch nach definierter Anzahl von ITerationen, Abbruch wenn nur noch kleine �nderungen beim Update erzielt werden oder Abbruch, die bis dato gelernter Strategie das Problem ann�hernd l�st.

Unter Verwendung der erlernten Q-Tabelle wird die beste Aktion f�r jeden Zustand basierend auf der Aktion mit dem h�chsten Q-Wert ausgew�hlt. So wird Q-learning durchgef�hrt.

\subsection{Steuerung des Lernverfahrens}

Abgesehen von der Anzahl der Epochen k�nnen die in der Q-Update-Regel enthaltenen Parameter \(\alpha\) und \(\alpha\) den Lernprozess beeinflussen.

\paragraph{Lernrate, \(\alpha\)}

Der Wert von \(\alpha\) muss zwischen 1 und 0 liegen. Ein zu gro�er Wert (\(\alpha \sim 1\)) k�nnte dazu f�hren, dass die Q-Funktion zu grob angen�hert wird. W�hlen Sie einen zu kleinen Alpha-Wert, (\(\alpha \sim 0\)), dann ben�tigt der Trainingsprozess mehr Zeit.

\paragraph{Discountfaktor, \(\gamma\)}

Der Wert von \(\gamma\) steuert, wie wichtig die Vorausschau auf das Ergebnis der Regel ist. Ohne Look-Ahead w�rden nur lokale Bewertungen ber�cksichtigt und keine globale Strategie gelernt. Daher wird normalerweise ein Wert von 0,8 bis 0,95 gew�hlt. Wenn der Wert jedoch zu gro� ist, kann dies den Belohnungswert �berschatten, was nicht erw�nscht ist.

\subsection{Rolle des Zufalls beim Q-Learning}

Bei der erstmaligen Initialisierung der Werte der Q-Tabelle kann es vorteilhaft sein, diese auf Zufallswerte zu initialisieren. Auf diese Weise kann es sein, dass einige der Werte bereits nahe am idealen Wert der Ann�herung liegen. Alternativ werden die Werte alle mit demselben festen Wert initialisiert.

In der Q-learn-Regel soll der Term \(max_k\{Q(s_{i+k}, a_k)\}\) die am besten bewertete Aktion im Vorausschauen ausw�hlen. Dies ist als ``greedy''-Ansatz bekannt und kann dazu f�hren, dass bessere globale Strategien �bersehen werden. Um dies zu �berwinden, wird \(\epsilon\textnormal{-soft}\) verwendet. Das bedeutet, dass in 1 bis \(\epsilon\) \% der F�lle, die gew�hlte Look-Ahead-Aktion zuf�llig ausgew�hlt wird.

\subsection{Bewertung von Q-Learning}