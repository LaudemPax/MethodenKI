\section{Lernen durch positives und negatives Feedback}

Das allgemeine Ziel eines Modells, das diese Art des Lernens durchl�uft, ist, dass es entweder ``�berlebt'' oder von erfolgreichen Reaktionen gewinnen. Das genauere Ziel des Modales ist es, eine �berlebens- oder Gewinnstrategie (engl. Policy) zu entwickeln, um die erhaltenen Belohnungen zu maximieren.

Der Lerner wei� zun�chst nichts dar�ber, was vor ihm liegt und was ihn vor der Exploration erwartet. Der Lernende muss sich also entscheiden, ob er einen neuen Zustand erkunden oder in einen bereits bekannten Zustand �bergehen soll.

\paragraph{Lernstrategiebeispiel:}

\begin{enumerate}
    \item Lerner befindet sich in einem Zustand \(s_i\) in einem \textbf{diskreten} Zustandsraum \(\Sigma\).
    \item Der Lerner versucht dann, ein oder mehrere Ziele zu erreichen.
    \item Im Zustand \(s_i\) kann der Lernende aus einer Reihe von auszuf�hrenden Aktionen \(A=\{a_{i,1}, \ldots, a_{a,n}\}\) w�hlen.
    \item Von der Ausf�hrung der Aktion \(a_{i,k}\) bewegt sich der Lerner zum n�chsten Zustand.
    \item Im n�chsten Zustand erh�lt der Lerner eine Belohnung oder Bestrafung, je nachdem, wie gut die Handlung dazu beigetragen hat, das Ziel zu erreichen.
\end{enumerate}

Dies wirft nat�rlich die Frage auf: Wie soll das Modell im Zustand \(s_i\) die beste Aktion, \(a\) ausw�hlen? In diesem Fall wird nach der Funktion \(f^\pi\) gesucht, die basierend auf dem aktuellen Zustand \(s_i\) die beste Aktion \(a_{i,k}\) empfiehlt: \[f^\pi:\Sigma \rightarrow A \textnormal{ mit } f^\pi(s_i)=a_{i,k} \]

In dieser Notation stellt \(\pi\) eine Abfolge von Aktionen (Strategie oder Policy) dar, die mit \(f^\pi\) bestimmt wird. \(f^\pi\) kann als eine Lookup-Tabelle implementiert werden. 

Ein alternativer Ansatz w�re, die Erfolgsaussichten der Durchf�hrung einer Aktion \(a_{i,k}\) in Zustand \(s_i\) zu bestimmen. Daf�r gibt es zwei Varianten:

\begin{itemize}
    \item \textbf{Value-Funktion:} \(V^\pi(s) \rightarrow w \in IR\). Wenn die Policy \(\pi\) im Zustand \(s\) befolgt wird, wird die Belohnung \(w\) erhalten.
    \item \textbf{Q-Funktion:} \(Q^\pi(s,a) \rightarrow w \in IR\). Wenn die Policy \(\pi\) im Zustand \(s\) befolgt wird, wird die Belohnung \(w\) erhalten. Der Vorteil des Q-Wertes ist, dass er als Entscheidungshilfe dient, um auszuw�hlen, welche verf�gbare Aktion, \(a_{i,k}\) im Zustand \(s_i\).
\end{itemize}