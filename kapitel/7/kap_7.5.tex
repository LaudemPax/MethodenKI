\section{Aufgaben bei der Entwicklung eines Expertensystems}

Um ein Expertensystem zu entwickeln, m�ssen bestimmte Schritte unternommen werden. An erster Stelle steht die Durchf�hrung einer Machbarkeitsanalyse. Expertensysteme sind nur in bestimmten Problemf�llen anwendbar, daher ist es wichtig, sich zu vergewissern, dass ein Expertensystem der richtige Methode ist. 

Danach ist es wichtig zu entscheiden, wie mit dem System interagiert werden soll. Das bedeutet, dass festgelegt werden muss, welche Aufgaben unterst�tzt werden sollen und welche Art von Dialog zwischen Benutzern und Experten gef�hrt werden muss.

Sobald dies erreicht ist, ist der n�chste Schritt der Wissenserwerb. Es m�ssen Experten gefunden und ihr Wissen gesammelt werden. Hier ist es wichtig, die richtigen Fragen zu stellen, um das gegebene Problem zu l�sen.

Als N�chstes folgt die Umsetzung des Wissens in Regeln, die Implementierung der Regeln in ein wissensbasiertes System und dann das Testen.