\section{Aufgaben bei der Entwicklung eines Expertensystems}

Um ein Expertensystem zu entwickeln, m�ssen bestimmte Schritte unternommen werden. An erster Stelle steht die Durchf�hrung einer Machbarkeitsanalyse. Expertensysteme sind nur in bestimmten Problemf�llen anwendbar, daher ist es wichtig, sich zu vergewissern, dass ein Expertensystem der richtige Methode ist. 

Danach ist es wichtig zu entscheiden, wie mit dem System interagiert werden soll. Das bedeutet, dass festgelegt werden muss, welche Aufgaben unterst�tzt werden sollen und welche Art von Dialog zwischen Benutzern und Experten gef�hrt werden muss.

Sobald dies erreicht ist, ist der n�chste Schritt der Wissenserwerb. Es m�ssen Experten gefunden und ihr Wissen gesammelt werden. Hier ist es wichtig, die richtigen Fragen zu stellen, um das gegebene Problem zu l�sen.

Als N�chstes folgt die Umsetzung des Wissens in Regeln, die Implementierung der Regeln in ein wissensbasiertes System und dann das Testen.

\section{Vertiefung: IBM Watson}

IBM Watson ist ein Frage-Antwort-Computersystem, das in der Lage ist, in nat�rlicher Sprache gestellte Fragen zu beantworten. Das Computersystem wurde urspr�nglich entwickelt, um Fragen in der Quizshow ``Jeopardy'' zu beantworten, und 2011 trat Watson gegen zwei Jeopardy-Champions an, um den ersten Platz zu gewinnen.

Watson funktioniert, indem es viele Expertensysteme anwendet, um zusammenzuarbeiten, um verschiedene Arten von Informationen zu verarbeiten. IBM hat angegeben, dass Watson mehr als 100 verschiedene Techniken verwendet, um nat�rliche Sprache zu analysieren, Quellen zu identifizieren, Hypothesen zu finden und zu generieren.

Laut David Ferucci, dem Leiter des IBM-Watson-Forschungsteams, wies die interne Infrastruktur von Watson als das hervor, was es von anderen Computersystemen unterscheidet\cite{IBMwatson}. Mit seiner Architektur ist Watson in der Lage, viele verschiedene Algorithmen gleichzeitig auszuf�hren, um Vertrauen in eine endg�ltige Antwort aufzubauen.