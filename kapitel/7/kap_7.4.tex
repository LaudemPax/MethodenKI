\section{Konfliktaufl�sung von Regelsystemen}

In vielen F�llen sind mehrere Regeln anwendbar, so dass sich die Frage stellt, welche Regeln man anwenden sollte. Die Antwort auf diese Frage h�ngt von der Reihenfolge der Regeln in der Regelbasis ab. In kommutativen Systemen kann eine schlechte Wahl der Regeln zu langen L�sungen f�hren. In nicht-kommutativen Systemen kann die Schwierigkeit der Suche aufgrund der R�ckverfolgung exponentiell sein. 

\paragraph{Ansatz 1: Heuristische Regelauswahl}

Wie bei den meisten Suchproblemen k�nnen Heuristiken den Prozess effizienter machen. Ein g�ngiger Ansatz besteht darin, sich an der H�ufigkeit der verwendeten Regel zu orientieren, d. h. jede Regel darf nur einmal verwendet werden. Ansonsten kann es von Vorteil sein, Regeln zu bevorzugen, die auf den neuesten generierten Daten beruhen. Schlie�lich ist es auch m�glich, eine Implementierung zu erstellen, die auf spezifischen Merkmalen des Problems basiert. Das bedeutet, dass Regeln, die mehr mit dem aktuellen Problem zu tun haben, bevorzugt werden.

\paragraph{Ansatz 2: Meta-Regeln}

Metaregeln sind Regeln �ber Regeln. In diesem Fall w�re ein Ansatz, Regeln �ber die Auswahl von Regeln in Form von Regeln zu bilden. Einige Meta-Regeln w�ren zum Beispiel: Regeln mit billigeren Verfahren gegen�ber teuren Verfahren bevorzugen, Regeln mit weniger gef�hrlichen Ans�tzen gegen�ber gef�hrlicheren Ans�tzen bevorzugen, Regeln von Experten gegen�ber Anf�ngern bevorzugen.