\chapter{Formaler Logik}

\section{Wissensrepr�sentation und wissensbasierte Verarbeitung auf der Grundlage formaler Logik}

Es gibt F�lle, in denen Schlussfolgerungen aus Fakten, Gegenst�nden oder Zusammenh�ngen gezogen werden m�ssen. Zum Beispiel manchmal Fragen wie: Hat Objekt A Merkmal B? Oder ist A mit B verwandt? m�ssen beantwortet werden.   Dies kann mit einem System geschehen, das formale Logik verwendet.

Die Idee, die hinter der Funktionsweise eines Logiksystems steht, ist die Definition der anwendungsspezifischen Fakten (durch eine Logiksprache) und die anschlie�ende Einspeisung dieser Definition in einen allgemeinen anwendungsunabh�ngigen Verarbeitungsmechanismus. Der Verarbeitungsmechanismus generiert dann eine L�sung auf der Grundlage der definierten Fakten. Dieser Verarbeitungsmechanismus erfolgt durch logische Schlussfolgerungen und wird auch als Inferenzmechanismus bezeichnet. 

\subsection{Syntax und Semantik einer Logiksprache}

Um Fakten und Zusammenh�nge zu definieren, ist eine formale Sprache notwendig. Die Syntax einer Logiksprache definiert, welche Zeichenketten korrekt ausgedr�ckt werden (auch bekannt als Terme, S�tze, Ausdr�cken oder Formeln).

Bei der Semantik geht es darum, wie die Sprache definiert, unter welchen Umst�nden ein Ausdruck wahr ist.  Semantik legt fest, wie die Sprache definiert, unter welchen Umst�nden ein Ausdruck in der gegebenen "Welt" wahr ist.

Die Aussagenlogik beschreibt Beziehungen zwischen Ausdr�cken, die wahr oder falsch sein k�nnen, und abstrahiert von der nat�rlichen Sprachstruktur der Aussage.

\paragraph{Syntax der Aussagenlogik}

Die in der formalen Sprache ausgedr�ckte Aussagenlogik hat eine lexikalische und eine sturkturelle Komponente. Die lexikalische Komponente besteht aus grundlegenden Symbolen wie Konstanten (TRUE, FALSE), Satzsymbolen (A, B, C), logischen Verkn�pfungssymbolen (\(\wedge \vee \neg \Leftrightarrow \Rightarrow\)) und Klammern. Die Strukturkomponente besteht aus Regeln, die syntaktisch korrekte S�tze (Aussagen) bilden. Ein Beispiel f�r einen Satz ist: \(A \vee \neg B \vee C \wedge D\) was bedeutet: ``A oder nicht B oder C und D''.

\paragraph{Semantik aussagenlogischer Ausdr�cke}

Die Semantik (d.h. die Bedeutung) eines aussagenlogischen Ausdrucks wird als eine Interpretation bezeichnet. Eine Interpretation ist eine Abbildung,\(I\) , die S�tze auf ihre Wahrheitswerte abbildet. 