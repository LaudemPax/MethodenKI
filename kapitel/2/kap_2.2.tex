\section{Probleml�sung mit KI}

\subsection{Schritte um Probleme zu l�sen}

\begin{enumerate}
    \item Zielformulierung:
    \begin{itemize}
        \item Soweit m�glich, Plausibilit�ts-Check dabei durchf�hren: Ist das Ziel machbar?
        \item \textbf{Beispiel:} Hans will von A nach B, kennt aber den Weg nicht.
    \end{itemize}
    \item Problemformulierung
    \begin{itemize}
        \item Ausgangssituation formulieren.
        \item feststellen welche Operationen m�glich sind (z.B Spielregeln).
        \item \textbf{Beispiel:} Durch ausf�hren von Fahr-Aktionen von A �ber verbundene Nachbarorte nach B kommen. M�gliche Operationen w�ren: in die benachbarten St�dte zu fahren.
    \end{itemize}
    \item Konstruktion einer L�sung
    \begin{itemize}
        \item bewerte G�te einer L�sung
        \item w�hle effektiven L�sungsweg
        \item \textbf{Beispiel:} Ein m�glicher Weg zur L�sung des Problems w�re die Erstellung eines Suchbaums.
    \end{itemize}
    \item Ausf�hrung
    \begin{itemize}
        \item L�uft alles wie geplant?
    \end{itemize}
\end{enumerate}

\subsection{Performanzma� berechnen}

\begin{itemize}
    \item Oft gibt es mehrere zul�ssige L�sungswege zu einem Probleme
    \item Wie findet man die optimalste L�sung?
    \item Zur Bewertung der \textbf{G�te} einer L�sung berechnet man die Gesamtkosten
\end{itemize}

\[Gesamtkosten=Suchkosten + Pfadkosten\]

\begin{itemize}
    \item Es ist oft schwierig, die G�te einer L�sung zu verrechnen, da es oft viele m�gliche Aspekte gibt, die beobachtet und gemessen werden k�nnen.
    \item Manchmal ist es besser, die weniger optimale L�sung zu w�hlen, die schneller berechnet werden kann: Genauere Planung kann mehr Zeit kosten als sie erspart!
\end{itemize}