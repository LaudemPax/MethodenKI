\section{Aufgaben beim Entwurf eines Wissenbassierten Systems}

Es ist es wichtig, von Anfang an herauszufinden, welche Arten von Wissen es zu erwerben gilt und wie diese repr�sentiert und gespeichert werden. Daneben muss auch entschieden werden, welche Inferenzmechanismen verwendet werden sollen.

Dann stellt sich die Frage, wie die Wissensbasis gef�llt werden soll. Dazu gibt es einfache Methoden, z. B. indem man einen Experten dazu bringt, sein Wissen mit einem Texteditor, einem Formular oder einem Mikrofon einzugeben. Es gibt aber auch komplexere Ans�tze wie die Entwicklung einer Schnittstelle zur Umwandlung von Datenbankinhalten in ein Format, das von einem wissensbasierten System verarbeitet werden kann, oder die Extraktion neuen Wissens mit Hilfe k�nstlicher neuronaler Netze. 

\section{Vertiefung: The Bitter Lesson}

In dem kurzen Essay von Richard Sutton, einem kanadischen Informatiker, diskutiert er den seiner Meinung nach gr��ten Fehler der KI-Forschung der letzten 70 Jahre. Er argumentiert, dass die KI-Forschung durch einen zu starken Fokus auf die Verwendung von Ans�tzen des ``menschlichen Wissens'' anstelle der Verwendung von Rechenleistung behindert wurde.

Als Beispiel verwies er auf das Jahr 1997, als Kasparov, Schachweltmeister, von einem Computer besiegt wurde, der Deep Search mit spezieller Hard- und Software verwendete, anstatt das menschliche Verst�ndnis der besonderen Natur des Schachs zu nutzen. Diese Methode wurde von der Mehrheit der Computerschachforscher abgelehnt, die entt�uscht waren, dass das System keine Methoden verwendete, die auf menschlichem Input basierten, um zu gewinnen.

�hnliche Situationen gab es bei der Entwicklung von Computer Go, Spracherkennung und Computer Vision. Dabei kamen zun�chst solche Ans�tze: die Besonderheiten des Spiels bei Go, die Kenntnis von W�rtern und Ph�nomenen f�r die Spracherkennung oder die Suche nach Kanten und Zylindern im Computer Vision. Am Ende wurde all dies jedoch zugunsten neuerer Methoden verworfen, die sich die rohe Rechenleistung zunutze machen, die aufgrund des Mooresschen Gesetzes zur Verf�gung gestellt wurde.

Die bitteren Lehren laut Richard Sutton sind, dass die Leistungsf�higkeit von Allzweckmethoden, die mit zunehmender Rechenleistung weiter skalieren, nicht untersch�tzt werden sollte und dass sich die KI-Forschung darauf konzentrieren sollte, KI-Agenten zu erm�glichen, herauszufinden, wie sie die Au�enwelt verstehen k�nnen. und der KI keine bestimmte Denkweise aufzuzwingen.