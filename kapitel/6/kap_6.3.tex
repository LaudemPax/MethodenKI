\section{Ans�tze zur Wissensverarbeitung}

Eine wichtige Komponente eines wissensbasierten Systems ist die Inferenzmaschine, die verwendet wird, um unsere Antworten auf Fragen zu finden. Es gibt viele Ans�tze f�r die Implementierung einer solchen Maschine zur Verarbeitung von vorhandenem Wissen.

\paragraph{Prinzip der Wissensabstraktion}

Diese Methode wird als induktive Inferenz bezeichnet: Ableitung allgemeiner Aussagen aus Spezialf�llen. z.B:

\begin{itemize}
    \item Beobachtung: Studenten Hans und Fritz haben ein Laptop
    \item Schlussfolgerung: Alle Studenten haben Laptops
\end{itemize}

\paragraph{Prinzip von Wissen �ber Wissensintegrit�t}

Diese Methode wird als deduktive Inferenz bezeichnet: Durch Erkennung von Regeln und Invarianzen aus formalen Beschreibungen wird explizites Wissen explizit gemacht. z.B:

\begin{itemize}
    \item Beobachtung: Aussage A ist wahr und Aussage B ist wahr.
    \item Schlussfolgerung: Auch die Aussage A oder B ist wahr.
\end{itemize}

\paragraph{Prinzip von Wissen �ber Kausalzusammenh�nge}

Diese Methode wird als abduktive Inferenz bezeichnet: Die Ursachen werden aus Beobachtungen abgeleitet: A ist der Grund daf�r, dass B passiert ist, wenn also B beobachtet wird, ist A passiert. z.B:

\begin{itemize}
    \item Beobachtung: Licht ist an.
    \item Schlussfolgerung: Lichtschalter ist auf Position ``Ein''.
\end{itemize}

\paragraph{Wissen �ber Analogien}

Diese Methode wird als analoge Inferenz bezeichnet: Neue Erkenntnisse werden aus bereits bekannten abgeleitet. z.B:

\begin{itemize}
    \item Beobachtung: durch ein dickes Rohr kann viel Wasser flie�en.
    \item Schlussfolgerung: durch ein dickes Kabel kann viel Strom flie�en.
\end{itemize}

\paragraph{Prinzip der Verwendung von empirischem Wissen}

Diese Methode wird als probabilistische Inferenz bezeichnet: Schlussfolgerung, dass eine Wirkung eintritt, basierend auf der Wahrscheinlichkeit, dass sie aufgrund der Situation eintritt. z.B:

\begin{itemize}
    \item Wenn 100 Personen im Raum sind, kann man davon ausgehen, dass mindestens zwei von ihnen denselben Geburtstag haben.
\end{itemize}

\paragraph{Prinzip der Verwendung von unscharfem Wissen}

Diese Methode wird als Fuzzy-Inferenz bezeichnet: Schlussfolgerungen auf der Grundlage der Wahrscheinlichkeit, dass ein Objekt zu einer Menge geh�rt. z.B:

\begin{itemize}
    \item Hans ist vierzig Jahre alt. Alte Menschen sind weise.
    \item Zugeh�rigkiet Hans zur Menge der jungen Menschen: 0.6
    \item Zugeh�rigkiet Hans zur Menge der alten Menschen: 0.4
    \item Ask: Ist Hans Weise? Antwort: Na ja, geht so
\end{itemize}